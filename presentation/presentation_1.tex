\documentclass[10pt]{beamer}
\let\captionsetup\relax 
\usepackage[hang,
			compatibility=false]{caption}
\usepackage[utf8x]{inputenc}
\usepackage[T1]{fontenc}
\usepackage[polish]{babel}
\usepackage{polski}
\usepackage{url}
\usepackage{hyperref}
\usepackage{listings}
\usepackage{caption}
\usepackage{minted}
\usepackage{color}

\usetheme{Marburg}
\usecolortheme{default}
\useoutertheme{shadow}
\useinnertheme{rounded}
\title[Mechanizm refleksji w języku C++]{Mechanizm refleksji w języku C++ wspierający mapowanie obiektowo-relacyjne}
\author[Tomasz Trębski 165535]{Tomasz Trębski}
\institute[Politechnika Łódzka :: FTIMS]{
  Politechnika Łódzka \\
  Wydział Fizyki Technicznej, Informatyki i Matematyki Stosowanej
}
\date[2013]{Łódź, 2013}

\AtBeginSection[]
{
  \frame{
    \frametitle{Agenda}
    \tableofcontents[currentsection, currentsubsection]
  }
}

\begin{document}
	\frame{
		\transboxin
		\titlepage
		\begin{center}
			\begin{columns}
				\column{0.1\textwidth}
				\includegraphics[scale=0.1]{img/ORM}
				\column{0.7\textwidth}
				\textbf{promotor:} dr inż. Arkadiusz Tomczyk
			\end{columns}
		\end{center}
	}
	\section{Problematyka pracy inżynierskiej}
  	\frame{
		\transboxin
   		\frametitle{Problematyka pracy inżynierskiej}
   		\framesubtitle{Czego praca ma dotyczyć, część praktyczna oraz teoretyczna}

		\begin{block}{Poruszana tematyka}
			\begin{itemize}
				\item biblioteki współdzielone jako narzędzie dystrybucji kodu,
				\item problem optymalizacji oraz zarządzania pamięcią (\textbf{implicit sharing} oraz \textbf{explicit sharing}),
				\item refleksja, a języki kompilowane bez wsparcia maszyn wirtualnych
				\item refleksja, jako wsparcie dla programów realizujących ORM
			\end{itemize}
		\end{block}
		\begin{block}{Cel pracy}
			\begin{itemize}
				\item zalety oraz wady refleksji w języku C++,
				\item metody przystosowania języka C++ do wspierania retrospekcji kodu,
				\item analiza istniejących rozwiązań problemu retrospekcji.
			\end{itemize}
		\end{block}
  	}
  	
  	\section{Istniejące rozwiązania wspierające refleksję}
  	\begin{frame}[fragile]
		\transboxin
    	\frametitle{Wykorzystanie refleksji w praktyce}
    	\framesubtitle{Kody źródłowe oraz opis}
    	\begin{columns}[c]
    		\column{.6\textwidth}
	    	\begin{block}{\includegraphics[scale=0.1]{img/java}}
				\inputminted[
					linenos=true,
					fontsize=\tiny,
		            numbersep=2pt
				]{java}{listings/java_reflection}
	    	\end{block}
	    	\begin{block}{\includegraphics[scale=0.1]{img/python}}
				\inputminted[
					linenos=true,
					fontsize=\tiny,
		            numbersep=2pt
				]{python}{listings/python_reflection}	
	    	\end{block}
	    	\begin{block}{\includegraphics[scale=0.1]{img/logo_js}}
				\inputminted[
					linenos=true,
					fontsize=\tiny,
		            numbersep=2pt
				]{javascript}{listings/js_reflection}	
	    	\end{block}
    		\column{.4\textwidth}
    		Cechą wspólną wszystkich tych rozwiązań jest
    		operowanie na instancja obiektów bez właściwej 
    		znajomości ich typu innej niż wiedza o nazwie
    		klasy bądź istnieniu konkretnej metody oraz działanie
    		wewnątrz dowolnie rozumianego wirtualnego środowiska. 
    	\end{columns}
	\end{frame}
  	
  	\section{Możliwe zastosowania refleksji}
  	\frame{
		\transboxin
    	\frametitle{Zastosowania refleksji}
    	\framesubtitle{O możliwych zastosowaniach}
    	\begin{enumerate}
    		\item \textbf{Paradygmat} - Programowanie zorientowane na refleksje \cite{ROP},
    		\item \textbf{Modyfikacja danych} - modyfikowanie obiektów bez właściwej znajomości ich typu,
    		jedynie na podstawie struktury wewnętrznej,
    		\item \textbf{Modyfikacja struktury} - dodawanie, usuwanie metod, zmiennych itp w czasie
    		wykonywania programu
    	\end{enumerate}
  	}
  	
  	\section{Harmonogram pracy}
  	\frame[allowframebreaks]{
		\transboxin
    	\frametitle{Plan pracy}
    	\begin{block}{Etap I}
	    	\begin{itemize}
	    		\item Zdefiniowania wymagań dla projektu.
	    		\item Wstępny projekt modelu klas.
	    		\item Prototyp funkcjonalności.
	    		\item Testy jednostkowe oraz dokumentacja.
	    		\item Część teoretyczna pracy inżynierskiej.
	    	\end{itemize}
    	\end{block}
    	\begin{block}{Etap II}
	    	\begin{itemize}
	    		\item Rozszerzenie projektu na wsparcie dla silnika ORM.
	    		\item Optymalizacja oraz dodanie wielowątkowości.
	    		\item Automatyczna notyfikacja o zmianach stanu modeli.
	    		\item Dalsza część teoretyczna pracy inżynierskiej.
	    	\end{itemize}
    	\end{block}
    	\begin{block}{Etap III}
	    	\begin{itemize}
	    		\item Optymalizacja zużycia pamięci: \textbf{implicit/explicit data sharing}.
	    		\item Opisanie części praktycznej: \textbf{wyniki, wnioski}.
	    	\end{itemize}
    	\end{block}
  	}
  	
  	\section{Technologie, wzorce, architektura}
  	\frame{
		\transboxin
    	\frametitle{Technologie, wzorce, architektura}
    	\framesubtitle{O używanych technologiach, bibliotekach itp.}
    	\begin{block}{\begin{center}\includegraphics[scale=0.06]{img/qt-logo}\end{center}}
    		Skorzystanie z szeregu gotowych metod oraz obiektów,
    		wspierających meta programowanie, EDP\footnote{Event-Driven-Programming} oraz 
    		podstawowe mechanizmy refleksji pozwalający na modyfikowanie obiektów
    	\end{block}
    	\begin{columns}
    		\column{.3\textwidth}
	    	\begin{block}{Wzorce projektowe}
	    		Użycie szeregu wzorców celem stworzenie aplikacji modularnej,
	    		skalowalnej i łatwiej w debugowaniu.
	    	\end{block}
	    	\column{.3\textwidth}
	    	\begin{block}{Język}
	    		Użycie wszystkich nowych dostępnych funkcji języka C++11
	    	\end{block}
	    	\column{.3\textwidth}
	    	\begin{block}{Architektura}
	    		Biblioteka współdzielona
	    	\end{block}
    	\end{columns}
  	}
  	\note{EDP - programowanie wspierające event oraz listenery}
  	
  	\section{Bibliografia}
  	\frame[allowframebreaks]{
		\transboxin
	 	\frametitle<presentation>{Z czego korzystałem do tej pory?}    
	 	\begin{thebibliography}{10}    
	  		\beamertemplatearticlebibitems
		  		\bibitem[Reflection-Oriented Programming]{ROP}
		    		Jonathan M. Sobel Daniel P. Friedman
		    	\newblock \em{Computer Science Department, Indiana University}.
		    	\newblock \href{http://www.cs.indiana.edu/~jsobel/rop.html}{Reflection-Oriented Programming}
	  		\beamertemplatearticlebibitems
		  		\bibitem[Reflection at Wiki]{Wiki_Reflection}
		  			Wikipedia Community
		    	\newblock \em{Reflection in computer programming}
		    	\newblock \href{http://en.wikipedia.org/wiki/Reflection_(computer_programming)}{Wiki - Computer reflection}.
	  		\beamertemplatearticlebibitems
		  		\bibitem[Hibernate Docs]{HIB_DOCS}
		  			Hibernate Community Materials
		    	\newblock \em{Hibernate Community Materials}
		    	\newblock \href{http://www.hibernate.org/docs}{Docs}
	  \end{thebibliography}
	}
	
	\section{Pytania ?}
  	\frame{
		\transboxin
    	\frametitle{Pytania ?}
    	\framesubtitle{Czas pytać....}
    	\begin{center}
    		\LARGE{Macie jakieś pytania?}
    		\includegraphics[scale=0.5]{img/question-mark}
    	\end{center}
  	}
\end{document}